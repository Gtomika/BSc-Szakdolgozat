\documentclass[]{beamer}
% Class options include: notes, notesonly, handout, trans,
%                        hidesubsections, shadesubsections,
%                        inrow, blue, red, grey, brown

% Theme for beamer presentation.
\usepackage{beamerthemesplit} 
\usepackage[utf8]{inputenc}
\usepackage{amsmath}
\usepackage{amssymb}
\usepackage{amsthm}
\usepackage{bm}
\usepackage{caption}

\usepackage{tikz}
\usetikzlibrary{positioning}

\newcommand{\bx}{\mathbf{x}}
\newcommand{\be}{\mathbf{e}}
\newcommand{\by}{\mathbf{y}}

\newtheorem{tetel}{Tétel}

\newcounter{definicioszam}
\newenvironment{definicio}[1]
{{\medskip}\noindent \stepcounter{definicioszam}
{\bfseries{\thedefinicioszam. Definíció: \textit{#1.}}}}{\bigskip }

\title{PageRank és kiszámolása}    % Enter your title between curly braces
\author{Gáspár Tamás}                 % Enter your name between curly braces
\date{2019 május 30.}

\begin{document}

% Creates title page of slide show using above information
\begin{frame}
  \titlepage
\end{frame}

\begin{frame}
  \frametitle{A PageRank definíciója}   % Insert frame title between curly braces

 A PageRank egy Sergei Brin és Larry Page által alkotott módszer arra, hogy weboldalak egy halmazát a köztük lévű linkek alapján rangsoroljuk.

 A weboldalak halmaza (röviden web) legyen egy irányított gráf, ahol az irányított élek fejezik ki az oldalak közti linkeket.
 
  \begin{figure}[h]
 	\begin{tikzpicture}[
 	roundnode/.style={circle, draw=green!60, fill=green!5, very thick, minimum size=7mm},
 	]
 	\centering
 	%Nodes
 	\node[roundnode]      (c1)                     {1};
 	\node[roundnode]      (c2)       [right=of c1] {2};
 	\node[roundnode]      (c3)       [below=of c2] {3};
 	\node[roundnode]      (c4)       [below=of c1] {4};
 	
 	%Lines
 	\draw[->,thick] (c1) -- (c2);
 	\draw[->,thick] (c2) -- (c3);
 	\draw[->,thick] (c4) -- (c2);
 	\draw[->,thick] (c4) -- (c3);
 	\end{tikzpicture}
 	\caption*{Egy 4 oldalból álló web}
 \end{figure}
 
\end{frame}

\begin{frame}
	\frametitle{A PageRank definíciója}
	
	\begin{definicio}{Weboldal PageRankja}
		Legyen adott egy web, ahol $V$ az oldalak halmaza. Legyen $v_i \in V$ oldal PageRankja $r(v_i)$, az oldalról kimenő linkek száma pedig $ |v_i| $. Jelölje $B_i \subset V$ azon oldalak halmazát amelyen linkelnek $v_i$-re.
		
		Ekkor $v_i$ oldal PageRank-ja definíció szerint:
		\[ r(v_i) = \sum_{v_j \in B_i} \frac{r(v_j)}{ |v_j| } \text{.} \]
	\end{definicio}

	Egy oldal fontossága azon múlik, hogy mennyi oldal linkel rá, és hogy ezek milyen fontosak.
	
	\textbf{Megjegyzés:} Rekurzív definíció.
\end{frame}

\begin{frame}
	\frametitle{A linkmátrix}
	
	Az oldalak között kapcsolatot mutatja.
	
	\begin{definicio}{Linkmátrix}
		\begin{equation*}
		a_{i,j}=\begin{cases}
		\frac{1}{ |v_i| }, & \text{ha $v_i$ linkel $v_j$-re,}\\
		0, & \text{egyébként}.
		\end{cases}
		\end{equation*}
	\end{definicio}

Egy példa: web és a hozzá tartozó linkmátrix.
\begin{minipage}{0.4\textwidth}
	\centering
	\begin{center}
		\begin{tikzpicture}[
		roundnode/.style={circle, draw=blue!60, fill=blue!5, very thick, minimum size=7mm},
		]
		%Nodes
		\node[roundnode]      (c1)                     {1};
		\node[roundnode]      (c2)       [right=of c1] {2};
		\node[roundnode]      (c3)       [below=of c2] {3};
		\node[roundnode]      (c4)       [below=of c1] {4};
		
		%Lines
		\draw[->,thick] (c1) -- (c3);
		\draw[->,thick] (c1) -- (c4);
		\draw[->,thick] (c2) -- (c3);
		\draw[->,thick] (c4) -- (c2);
		\draw[->,thick] (c3) -- (c1);
		\draw[->,thick] (c2) -- (c1);
		\end{tikzpicture}
	\end{center}
\end{minipage}
\begin{minipage}{0.5\textwidth}
	\begin{center}
		\[
		\begin{pmatrix}
		0 & 0 & \frac{1}{2} & \frac{1}{2} \\  
		\frac{1}{2} & 0 & \frac{1}{2} & 0 \\
		1 & 0 & 0 & 0 \\
		0 & 1 & 0 & 0 \\
		\end{pmatrix}
		\]
	\end{center}
\end{minipage}
\end{frame}

\begin{frame}
	\frametitle{Webhez rendelhető markov lánc}
	
	\textbf{Megjegyzés:} A linkmátrix sztochasztikus lesz (kivétel: ha van \textit{"lógó oldal"}, de ez egyszerű helyettesítéssel megszüntethető).
	
	\bigskip
	
	Minden webhez rendelhetünk egy Markov-láncot, melynek állapotai az oldalak, átmenetvalószínűségei pedig a linkmátrix megfelelő elemei (ez lesz az átmenetmátrix). 
	
	\bigskip
	
	A \textit{"véletlen szörföző"} egyenletes eloszlás szerint halad az oldalon lévő linkeken. 
\end{frame}

\begin{frame}
	\frametitle{Definíció és a linkmátrix kapcsolata, PageRank vektor}
	
	Kapcsolat van a linkmátrix és a definícióból kapott egyenletrendszer között. Legyen $\bx$ az oldalak PageRank-jaiból álló sorvektor, ekkor:
	\[ 
	\bx = \bx A, 
	\]
	ami átalakítva
	\[
	\bx^T = A^T \bx^T. 
	\]
	
	A fenti egyenlet mutatja, hogy a $\bx$ a webhez rendelt markov lánc invariáns eloszlása lesz, a lenti pedig azt, hogy az 1-hez
	tartozó sajátvektor is (az $A^T$ mátrixban).
	
	\bigskip
	\textbf{Megjegyzés:} $\bx$-et vegyük úgy, hogy a komponensek összege 1.
\end{frame}

\begin{frame}
	\frametitle{A hatványiteráció}
	
	Hogyan számoljuk ki $\bx$-et? A pontos eredmény nem kivitelezhető, túl sok oldal. A megoldás a hatványiteráció:
	\[ \bx^{i+1} = A^T \bx^i \]
	
	\textbf{Probléma:} A hatványiteráció a domináns sajátértékhez konvergál, ilyen nem biztos hogy van (például -1, vagy 1 többszörös multiplicitással). Különálló \textit{"szubwebek"} esetén fordul elő.
	
	\bigskip
	A linkmátrixon (és a Markov-láncon) módosítani kell úgy, hogy a domináns sajátérték garantált legyen.
\end{frame}

\begin{frame}
	\frametitle{A linkmátrix módosítása: Google mátrix}
	
	Megközelítés a Markov-láncok irányából: Ha a Markov-lánc irreducibilis és aperiodikus, akkor biztosan egyértelmű lesz az invariáns eloszlás.
	
	\begin{definicio}{Google mátrix}
		Legyen $A \in \mathbb{R}^{n \times n}$ egy web linkmátrixa, $S$ pedig egy egy olyan $n \times n$-es mátrix, melynek minden eleme $\frac{1}{n}$. Ekkor a Google mátrix definíció szerint
		\[ G := \alpha A + (1-\alpha) S, \qquad \alpha \in [0,1] \]
	\end{definicio}

\vspace{-0.8cm}
Példa: Egy linkmátrix és az $\alpha = 0,7$ paraméterrel kapott Google mátrix.
\[A =
\begin{pmatrix}
0 & \frac{1}{2} & \frac{1}{2} & 0 \\
\frac{1}{3} & 0 & \frac{1}{3} & \frac{1}{3} \\
\frac{1}{3} & \frac{1}{3} & 0 & \frac{1}{3} \\
0 & 0 & 1 & 0 \\
\end{pmatrix},
G = 
\begin{pmatrix}
0,\!038 & 0,\!463 & 0,\!463 & 0,\!038 \\
0,\!321 & 0,\!038 & 0,\!321 & 0,\!321 \\
0,\!321 & 0,\!321 & 0,\!38 & 0,\!321 \\
0,\!038 & 0,\!038 & 0,\!888 & 0,\!038 \\
\end{pmatrix}
\]
\end{frame}

\begin{frame}
	\frametitle{A Google mátrix}
	
	A Google mátrix szintén sztochasztikus, és a szükséges tulajdonságokat is teljesíti. A hatványiteráció ($G$ transzponáltján végezve) tehát mindig PageRank vektorhoz fog konvergálni.
	
	\bigskip
	
	\textbf{Kérdés:} A konvergencia sebessége. A második legnagyobb sajátértéktől ($\lambda_2$) függ. Minél közelebb van ez 1-hez, annál lassabb.
	
	\bigskip
	
	\textbf{Megjegyzés:} \textit{(A helyettesítés mögötti heurisztika.)} \newline
	A véletlen szörföző most már nem csak a linkeken keresztül juthat el a következő oldalra, hanem $1-\alpha$ valószínűséggel egy egyenletes eloszlás szerint választott véletlen oldalra ugrik.
\end{frame}

\begin{frame}
	\frametitle{Az $\alpha$ paraméter jelentősége}
	
	\[ G := \alpha A + (1-\alpha) S, \qquad \alpha \in [0,1] \]
	
	Minden Google mátrixra és $\alpha$-ra: $\lambda_2 \leq \alpha$ (bizonyos feltételek* mellett $\lambda_2 = \alpha$).
	
	\begin{itemize}
		\item Nagy $\alpha$ lelassítja konvergencia sebességet.
		\item Kis $\alpha$ elnyomja a linkek jelentőségét (a véletlen szörföző $1-\alpha$ valószínűséggel nem a linkeken halad tovább).
	\end{itemize}
	
	\bigskip
	\textit{*: $A$-ban van legalább 2 irreducibilis zárt részhalmaz.}
\end{frame}

\begin{frame}
	\frametitle{Algoritmus a PageRank vektor meghatározására}
	\begin{enumerate}
		\item Adjuk meg az ($n$ oldalból álló) webhez tartozó linkmátrixot.
		
		\item Végezzük el a helyettesítést a \textit{"lógó oldalak"} kiküszöbölésére, azaz ha a linkmátrixban van csupa nulla sor, azt cseréljük le $1/n$ elemekből álló sorra.
		
		\item Válasszunk egy $\alpha \in (0,1)$ paramétert (kompromisszum). Konstruáljuk meg a Google mátrixot ezzel a paraméterrel:
		\[G = \alpha A + (1-\alpha)S,\]
		
		\item Végezzük a hatványiterációt a $G^T$ mátrixon, amíg a két iterációs lépés közötti különbség nem csökken valami kicsi $\epsilon$ érték alá, vagy adjunk meg konkrét iterációs lépésszámot.
		
		\item Az iteráció eredménye a PageRank vektor közelítése, ahol az $i.$ komponens az $i.$ oldal fontosságát adja meg.
	\end{enumerate}
\end{frame}

\end{document}
